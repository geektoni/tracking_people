% This is samplepaper.tex, a sample chapter demonstrating the
% LLNCS macro package for Springer Computer Science proceedings;
% Version 2.20 of 2017/10/04
%
\documentclass[runningheads]{llncs}
%
\usepackage{graphicx}
% Used for displaying a sample figure. If possible, figure files should
% be included in EPS format.
%
% If you use the hyperref package, please uncomment the following line
% to display URLs in blue roman font according to Springer's eBook style:
% \renewcommand\UrlFont{\color{blue}\rmfamily}

\begin{document}
%
\title{Detecting and Tracking People Motion}
%
%\titlerunning{Abbreviated paper title}
% If the paper title is too long for the running head, you can set
% an abbreviated paper title here
%
\author{Giovanni De Toni (197184)\inst{1}}
%
%\authorrunning{F. Author et al.}
% First names are abbreviated in the running head.
% If there are more than two authors, 'et al.' is used.
%
\institute{University of Trento, Italy, \email{giovanni.detoni@studenti.unitn.it}}
%
\maketitle              % typeset the header of the contribution
%
\begin{abstract}
The abstract should briefly summarize the contents of the paper in
150--250 words.

\keywords{Tracking \and Detection  \and MOG \and Background Subtraction \and Kalman Filter \and OpenCV}
\end{abstract}
%
%
%
\section{Introduction}
People detection and their subsequent tracking is a relatively easy task for humans, however, when dealing with a computer video system things becomes much more complicated. Computers and video cameras are not real people and they do not possess our capability to easily recognize objects and subsequent track their motion.  The need to recognize and follow people by just looking at a video recording has become one important task which has many real-world applications. We can enumerate a few of them: surveillance systems, autonomous vehicles, etc. 

\section{Related Works}

\section{Methods}

The detection/tracking process was divided into several steps, each of them dealing with a specific issue/task. Moreover, the procedure was devised in order to use simple algorithms in order to provide a relatively easy-to-use system that requires a moderate computational power in order to be applied to real-time systems. The processing pipeline works as follow:
\begin{enumerate}
\item Shadow removal and Background Subtraction;
\item Feature extraction from the selected contours (centroids+histogram);
\item Prediction of their position/motion by using the Kalman Filter;
\end{enumerate}

\subsection{Objects (Humans) Detection}
The detection of motion and of the pedestrian was composed by two separate tasks: shadow removal and background subtraction.
At the end of the procedure, for each frame, a set of contours $C$ was generated. These contours represented the moving object detected
by the algorithm in that specific video frame. 

\subsubsection{Shadow Removal}

Shadows are defined as a variation of luminance of a particular area of a specific video frame. They are a source of trouble when we need to detect the motion of objects in a video recording. A shadow usually cannot be distinguished from the real object casting it and therefore it is recognized as a moving part of the video. Intelligent shadows detection and removal is still an ongoing research topic since there is no way to perfectly avoid them.  

\subsubsection{People Detection}
\subsection{People Tracking}
\section{Experiments and Results}
\section{Conclusions}

%
% the environments 'definition', 'lemma', 'proposition', 'corollary',
% 'remark', and 'example' are defined in the LLNCS documentclass as well.
%
\begin{proof}
Proofs, examples, and remarks have the initial word in italics,
while the following text appears in normal font.
\end{proof}
For citations of references, we prefer the use of square brackets
and consecutive numbers. Citations using labels or the author/year
convention are also acceptable. The following bibliography provides
a sample reference list with entries for journal
articles~\cite{ref_article1}, an LNCS chapter~\cite{ref_lncs1}, a
book~\cite{ref_book1}, proceedings without editors~\cite{ref_proc1},
and a homepage~\cite{ref_url1}. Multiple citations are grouped
\cite{ref_article1,ref_lncs1,ref_book1},
\cite{ref_article1,ref_book1,ref_proc1,ref_url1}.
%
% ---- Bibliography ----
%
% BibTeX users should specify bibliography style 'splncs04'.
% References will then be sorted and formatted in the correct style.
%
% \bibliographystyle{splncs04}
% \bibliography{mybibliography}
%
\begin{thebibliography}{8}
\bibitem{ref_article1}
Author, F.: Article title. Journal \textbf{2}(5), 99--110 (2016)

\bibitem{ref_lncs1}
Author, F., Author, S.: Title of a proceedings paper. In: Editor,
F., Editor, S. (eds.) CONFERENCE 2016, LNCS, vol. 9999, pp. 1--13.
Springer, Heidelberg (2016). \doi{10.10007/1234567890}

\bibitem{ref_book1}
Author, F., Author, S., Author, T.: Book title. 2nd edn. Publisher,
Location (1999)

\bibitem{ref_proc1}
Author, A.-B.: Contribution title. In: 9th International Proceedings
on Proceedings, pp. 1--2. Publisher, Location (2010)

\bibitem{ref_url1}
LNCS Homepage, \url{http://www.springer.com/lncs}. Last accessed 4
Oct 2017
\end{thebibliography}
\end{document}
